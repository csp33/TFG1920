
\cleardoublepage
\thispagestyle{empty}

\begin{center}
{\large\bfseries  HOW-R-U?: Suite of e-coaches aimed to analyse human behaviour}\\
\end{center}
\begin{center}
Carlos Sánchez Páez\\
\end{center}

%\vspace{0.7cm}
\noindent{\textbf{Palabras clave}: chatbot, telegram, salud, médico, asistente, coach, suite}\\

\vspace{0.7cm}
\noindent{\textbf{Resumen}}\\

Hoy en día los trastornos mentales siguen siendo difíciles de tratar y diagnosticar. Además, debido al estigma que conllevan, algunos de los posibles pacientes no se sienten cómodos acudiendo a un profesional de la psicología. El objetivo de este proyecto es desarrollar una suite de e-coaches en forma de chatbots, específicamente, un chatbot de salud mental que ayude al diagnóstico prematuro de enfermedades mentales como la ansiedad y la depresión. Esta suite será modular, de forma que cada especialista pueda tener un agente conversacional asociado (psicólogo, nutricionista, etc.). Además, se le incorporará funcionalidad para que sea útil no solo para doctores, sino para analistas de datos, que contarán con la información recopilada de todos los pacientes.\\

Se plantea como trabajo futuro el uso de este sistema en un entorno en el que el asistente iniciará una conversación regularmente con el paciente y le hará una serie de preguntas definidas por su doctor. Tras ello, el especialista podrá acceder a una interfaz web en la que consultará y analizará las distintas respuestas proporcionadas por el paciente.
\cleardoublepage


\thispagestyle{empty}


\begin{center}
{\large\bfseries HOW-R-U?: Suite of e-coaches aimed to analyse human behaviour}\\
\end{center}
\begin{center}
Carlos Sánchez Páez\\
\end{center}

%\vspace{0.7cm}
\noindent{\textbf{Keywords}: chatbot, telegram, health, doctor, assistant, coach, suite}\\

\vspace{0.7cm}
\noindent{\textbf{Abstract}}\\

Nowadays mental ilnesses are still difficult to be treated and diagnosed. Moreover, they are associated to a stigma, so some possible patients do not feel comfortable when requesting profesional help. The aim of this project is to develop an e-coaches suite as chatbots, specifically a mental health chatbot
that would help to the premature diagnostic of mental illnesses such as anxiety and depression. This suite will be modular so that every specialist can have an associated conversational agent (psychologist, nutritionist, etc.). Moreover, it will also support data analysts, special doctors that will be able to consult all the information from all the patients.\\

It is proposed the use of this system in an environment where the assistant will regulary start a conversation with the patients, making them questions defined by their doctors. After that, the specialists will be able to access a web interface where they can consult and analyze the answers given by patients.

\newpage

\section*{}
\thispagestyle{empty}

\noindent\rule[-1ex]{\textwidth}{2pt}\\[4.5ex]

Yo, \textbf{Carlos Sánchez Páez}, alumno de la titulación Graduado en Ingeniería Informática de la \textbf{Escuela Técnica Superior
de Ingenierías Informática y de Telecomunicación de la Universidad de Granada}, con DNI 25613096C
, autorizo la
ubicación de la siguiente copia de mi Trabajo Fin de Grado en la biblioteca del centro para que pueda ser
consultada por las personas que lo deseen.

\vspace{6cm}

\begin{center}
  Fdo: Carlos Sánchez Páez

\end{center}

\vspace{2cm}

\begin{flushright}
Granada a 01 de julio de 2020
\end{flushright}

\newpage

\section*{}
\thispagestyle{empty}

\noindent\rule[-1ex]{\textwidth}{2pt}\\[4.5ex]

D. \textbf{Oresti Baños Legrán}, Profesor del Departamento Arquitectura de Computadores de la Universidad de Granada.



\vspace{0.5cm}

\textbf{Informa:}

\vspace{0.5cm}

Que el presente trabajo, titulado \textit{\textbf{HOW-R-U?: Suite of e-coaches aimed to analyse human behaviour}},
ha sido realizado bajo su supervisión por \textbf{Carlos Sánchez Páez}, y autorizo la defensa de dicho trabajo ante el tribunal
que corresponda.

\vspace{0.5cm}

Y para que conste, expide y firma el presente informe en Granada a 01 de julio de 2020

\vspace{1cm}

\textbf{El supervisor:}

\vspace{5cm}
\begin{center}
\textbf{Oresti Baños Legrán}

\end{center}


\newpage

\section*{Agradecimientos}
\thispagestyle{empty}

\vspace{1cm}
\begin{flushleft}
A mi familia, porque sin ellos nunca habría llegado tan lejos.

A Cristina, por su apoyo incondicional en todo momento.

A Casandra, por haber estado conmigo desde el principio.

A Oresti, por haberme apoyado en momentos difíciles y ser una gran fuente de motivación y conocimiento.

Al los buenos profesores que he encontrado durante mi carrera académica, por haber sabido enseñar y motivar.

Al mis amigos y amigas, por haberme ayudado siempre que lo he necesitado.
\end{flushleft}
