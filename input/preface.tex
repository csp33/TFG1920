
\cleardoublepage
\thispagestyle{empty}

\begin{center}
{\large\bfseries HOW-R-U?: Analising chatbot messages to automatically infer human behaviour}\\
\end{center}
\begin{center}
Carlos Sánchez Páez\\
\end{center}

%\vspace{0.7cm}
\noindent{\textbf{Palabras clave}: chatbot, telegram, salud, médico, asistente, coach, suite}\\

\vspace{0.7cm}
\noindent{\textbf{Resumen}}\\

Hoy día los trastornos mentales siguen siendo difíciles de tratar y diagnosticar. Además, debido al estigma que conllevan, los posibles pacientes
no se sienten cómodos acudiendo a un profesional de la psicología. El objetivo de este proyecto es desarrollar una suite de e-coaches en forma de
chatbots, específicamente, un chatbot de salud mental que ayude al prematuro diagnóstico de enfermedades mentales como la ansiedad y la depresión.\\

El asistente iniciará una conversación regularmente con el usuario y tomará datos según nuestras respuestas. Tras ello, se enviará un informe a su
médico para que éste realice un diagnóstico.
\cleardoublepage


\thispagestyle{empty}


\begin{center}
{\large\bfseries HOW-R-U?: Analising chatbot messages to automatically infer human behaviour}\\
\end{center}
\begin{center}
Carlos Sánchez Páez\\
\end{center}

%\vspace{0.7cm}
\noindent{\textbf{Keywords}: chatbot, telegram, health, doctor, assistant, coach, suite}\\

\vspace{0.7cm}
\noindent{\textbf{Abstract}}\\

Nowadays mental ilnesses are still difficult to be treated and diagnosed. Moreover, they are associated to a stigma, so possible patients do not feel
comfortable when requesting profesional help. The aim of this project is to develop an e-coaches suite as chatbots, specifically a mental health chatbot
that would help to the premature diagnostic of mental illnesses such as anxiety and depression.\\

The assistant will establish a conversation with the user in a regular way. It will take data according to the responses and send it to his doctor so that
he can perform a diagnostic.

\newpage

\section*{}
\thispagestyle{empty}

\noindent\rule[-1ex]{\textwidth}{2pt}\\[4.5ex]

Yo, \textbf{Carlos Sánchez Páez}, alumno de la titulación Graduado en Ingeniería Informática de la \textbf{Escuela Técnica Superior
de Ingenierías Informática y de Telecomunicación de la Universidad de Granada}, con DNI 25613096C
, autorizo la
ubicación de la siguiente copia de mi Trabajo Fin de Grado en la biblioteca del centro para que pueda ser
consultada por las personas que lo deseen.

\vspace{6cm}

\begin{center}
  Fdo: Carlos Sánchez Páez

\end{center}

\vspace{2cm}

\begin{flushright}
Granada a X de mes de 2020 .
\end{flushright}

\newpage

\section*{}
\thispagestyle{empty}

\noindent\rule[-1ex]{\textwidth}{2pt}\\[4.5ex]

D. \textbf{Oresti Baños Legrán}, Profesor del Departamento Arquitectura de Computadores de la Universidad de Granada.



\vspace{0.5cm}

\textbf{Informa:}

\vspace{0.5cm}

Que el presente trabajo, titulado \textit{\textbf{HOW-R-U?: Analising chatbot messages to automatically infer human behaviour}},
ha sido realizado bajo su supervisión por \textbf{Carlos Sánchez Páez}, y autorizo la defensa de dicho trabajo ante el tribunal
que corresponda.

\vspace{0.5cm}

Y para que conste, expide y firma el presente informe en Granada a X de mes de 2020 .

\vspace{1cm}

\textbf{El supervisor:}

\vspace{5cm}
\begin{center}
\textbf{Oresti Baños Legrán}

\end{center}


\newpage

\section*{Agradecimientos}
\thispagestyle{empty}

       \vspace{1cm}


AGRADECIMIENTOS
