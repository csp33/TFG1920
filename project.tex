\documentclass[12pt,english]{article}
\usepackage[a4paper,left=3cm,right=2cm,top=2.5cm,bottom=2.5cm]{geometry}
\usepackage[utf8]{inputenc}
\usepackage[english]{babel}
\usepackage{graphicx}
\usepackage{color}
\usepackage{xcolor}
\usepackage{colortbl}
\usepackage{amsthm,thmtools}
\usepackage{multirow}
\usepackage{amsmath}
\usepackage{subcaption}
\usepackage{adjustbox}
\usepackage{multirow}
\usepackage[hidelinks]{hyperref}
\usepackage{caption}
\usepackage{apacite}
\usepackage{amsthm}
\usepackage{multicol}
\usepackage{float}
\usepackage{amsfonts}
\usepackage{titling}
\usepackage{soul}
\usepackage{pgfplots}
\usepackage[nottoc]{tocbibind}
\usepackage{listings}
\usepackage{array}
\usepackage[framemethod=tikz]{mdframed}

\graphicspath{ {./img/}}
\makeatletter
\def\input@path{{input/}}
\makeatother
\selectlanguage{english}


\lstset{
  breaklines=true,
  postbreak=\mbox{\textcolor{red}{$\hookrightarrow$}\space},
}


\makeindex

\definecolor{light-gray}{gray}{0.95}
\lstset{columns=fullflexible,basicstyle=\ttfamily}
\surroundwithmdframed[
  hidealllines=true,
  backgroundcolor=light-gray,
  innerleftmargin=0pt,
  innertopmargin=0pt,
  innerbottommargin=0pt]{lstlisting}

\pgfplotsset{width=8cm,compat=1.9, xlabel={Year},
  ylabel={Number of documents}, xtick distance={2},
  ytick distance={2}, ymajorgrids=true,grid style=dashed,
  /pgf/number format/.cd,use comma,1000 sep={}}


\begin{document}

\begin{titlepage}

 \newlength{\centeroffset}
 \setlength{\centeroffset}{-0.5\oddsidemargin}
 \addtolength{\centeroffset}{0.5\evensidemargin}
 \thispagestyle{empty}

 \noindent\hspace*{\centeroffset}
 \begin{minipage}{\textwidth}

  \centering
  \includegraphics[width=0.9\textwidth]{logo_ugr.jpg}\\[1.4cm]

  \textsc{ \Large Bachelor Final Project\\[0.2cm]}
  \textsc{Computer Engineering}\\[1cm]

  {\Huge\bfseries HOW-R-U?\\}
  \noindent\rule[-1ex]{\textwidth}{3pt}\\[3.5ex]
  {\large\bfseries Suite of e-coaches aimed to analyse human behaviour}
 \end{minipage}

 \vspace{1cm}
 \noindent\hspace*{\centeroffset}
 \begin{minipage}{\textwidth}
  \centering

  \textbf{Author}\\ {Carlos Sánchez Páez}\\[2.5ex]
  \textbf{Supervisor}\\
  {Oresti Baños Legrán}\\[3ex]
  \includegraphics[width=0.4\textwidth]{etsiit_logo.png}\\[0.1cm]
  \vspace{2.5ex}
  \includegraphics[width=0.15\textwidth]{atc.jpg}\\[0.1cm]
  \vspace{1cm}
  \textsc{Escuela Técnica Superior de Ingenierías Informática y de Telecomunicación}\\
  \vspace{1cm}
  \textsc{Granada, academic year 2019-2020}
 \end{minipage}
\end{titlepage}


\cleardoublepage
\thispagestyle{empty}

\begin{center}
{\large\bfseries HOW-R-U?: Analising chatbot messages to automatically infer human behaviour}\\
\end{center}
\begin{center}
Carlos Sánchez Páez\\
\end{center}

%\vspace{0.7cm}
\noindent{\textbf{Palabras clave}: chatbot, telegram, salud, médico, asistente, coach, suite}\\

\vspace{0.7cm}
\noindent{\textbf{Resumen}}\\

Hoy día los trastornos mentales siguen siendo difíciles de tratar y diagnosticar. Además, debido al estigma que conllevan, los posibles pacientes
no se sienten cómodos acudiendo a un profesional de la psicología. El objetivo de este proyecto es desarrollar una suite de e-coaches en forma de
chatbots, específicamente, un chatbot de salud mental que ayude al prematuro diagnóstico de enfermedades mentales como la ansiedad y la depresión.\\

El asistente iniciará una conversación regularmente con el usuario y tomará datos según nuestras respuestas. Tras ello, se enviará un informe a su
médico para que éste realice un diagnóstico.
\cleardoublepage


\thispagestyle{empty}


\begin{center}
{\large\bfseries HOW-R-U?: Analising chatbot messages to automatically infer human behaviour}\\
\end{center}
\begin{center}
Carlos Sánchez Páez\\
\end{center}

%\vspace{0.7cm}
\noindent{\textbf{Keywords}: chatbot, telegram, health, doctor, assistant, coach, suite}\\

\vspace{0.7cm}
\noindent{\textbf{Abstract}}\\

Nowadays mental ilnesses are still difficult to be treated and diagnosed. Moreover, they are associated to a stigma, so possible patients do not feel
comfortable when requesting profesional help. The aim of this project is to develop an e-coaches suite as chatbots, specifically a mental health chatbot
that would help to the premature diagnostic of mental illnesses such as anxiety and depression.\\

The assistant will establish a conversation with the user in a regular way. It will take data according to the responses and send it to his doctor so that
he can perform a diagnostic.

\newpage

\section*{}
\thispagestyle{empty}

\noindent\rule[-1ex]{\textwidth}{2pt}\\[4.5ex]

Yo, \textbf{Carlos Sánchez Páez}, alumno de la titulación Graduado en Ingeniería Informática de la \textbf{Escuela Técnica Superior
de Ingenierías Informática y de Telecomunicación de la Universidad de Granada}, con DNI \input{id.tex}, autorizo la
ubicación de la siguiente copia de mi Trabajo Fin de Grado en la biblioteca del centro para que pueda ser
consultada por las personas que lo deseen.

\vspace{6cm}

\begin{center}
  Fdo: Carlos Sánchez Páez

\end{center}

\vspace{2cm}

\begin{flushright}
Granada a X de mes de 2020 .
\end{flushright}

\newpage

\section*{}
\thispagestyle{empty}

\noindent\rule[-1ex]{\textwidth}{2pt}\\[4.5ex]

D. \textbf{Oresti Baños Legrán}, Profesor del Departamento Arquitectura de Computadores de la Universidad de Granada.



\vspace{0.5cm}

\textbf{Informa:}

\vspace{0.5cm}

Que el presente trabajo, titulado \textit{\textbf{HOW-R-U?: Analising chatbot messages to automatically infer human behaviour}},
ha sido realizado bajo su supervisión por \textbf{Carlos Sánchez Páez}, y autorizo la defensa de dicho trabajo ante el tribunal
que corresponda.

\vspace{0.5cm}

Y para que conste, expide y firma el presente informe en Granada a X de mes de 2020 .

\vspace{1cm}

\textbf{El supervisor:}

\vspace{5cm}
\begin{center}
\textbf{Oresti Baños Legrán}

\end{center}


\newpage

\section*{Agradecimientos}
\thispagestyle{empty}

       \vspace{1cm}


AGRADECIMIENTOS

\thispagestyle{empty}
\newpage
\tableofcontents{}
\newpage
\listoffigures
\thispagestyle{empty}
\newpage

\section{Introduction}

The aim of this project is to create a Telegram bot that will act as a personal couch, specifically a psychologist.\\

In addition, it is intended to develop a software development kit to easily create another coaches (for example a nutritionist). All these couches will interact with a Natural Processing Language module that will transform the user response into an ontology, so that the bot can naturally continue the conversation as it takes relevant data to be shown to an specialist.

\subsection{Context}
Mental disorders are very common in our society. They are difficult to be diagnosed and properly treated. Most intervention programs do not last as much as they should and doctors have a very high workload, so patients need to wait for a long time before being advised by a doctor. In addition, although having a mental disease is very common, it is still a taboo subject whose stigma makes them even more difficult to be diagnosed and treated.\\

In addition, there is not a continuous traceability of patients' health status, so relevant data is not retrieved.



\subsection{Motivation}


On the other hand, smartphones are becoming increasingly integrated in our lives. Chat applications are used on a daily basis.\\

Modern chat applications allow us to create bots that can interact with us like if there was a person on the other side.\\

As we previously discussed, mental disorders are taboo, so having a bot you can talk with about how was your day, feelings, etc. could lead to an easier way of diagnosing them because chat conversations are seen as ''natural'' by the society. In addition, the bot will chat with us in a natural way thanks to the NLP module. This will help the patient not to consider the coach a doctor, making easier to tell it how he/she feels.

\subsection{Objectives}

MOSCOW ?
- parse audios, emojis, etc
- different languages ?
- architecture (containers)

\subsection{Structure}

\newpage
\subsection{State of the art}

\subsubsection{Classification of conversational agents}

According to \cite{Montenegro201956}, we can classify chatbots according to the following items:

\begin{figure}[H]
  \centering
  \includegraphics{taxonomy.jpg}
  \caption{Classification of conversational agents}
\end{figure}


\subsubsection{Health conversational agents to teach}

Chatbots can help students by creating a simulation of a particular scenario so that they can learn how to act.
\begin{itemize}
  \item \cite{Lopez2008194} offers a simulation of a real patient that presents several symptoms. Students must interview him and make a diagnostic as a doctor would do in real life.
  \item \cite{Shorey2019e14658} intends to improve undergraduates communication skills with patients.
\end{itemize}


\subsubsection{Health conversational agents to help}

Another useful field of application is helping professionals to do their work.

\begin{itemize}
  \item \cite{Ni201738} is an agent that interviews patients before the doctor does. Mandy elaborates a diagnostic based on several questions about patients symptoms and sends it to the doctor so that he can save that time.
  \item As professional treatments do not last as much, complements such as \cite{DAlfonso2017} appear. They offer a long-term treatment after the professional one so that the patient's progress is not lost.
\end{itemize}

\subsubsection{Effectiveness of health conversational agents}

Health coaches can be really helpful, as we can see in the following publications:

\begin{itemize}
  \item \cite{Breso2016297} proposes a platform to identify and provide early intervention for symptoms of depression and suicide. Its usability percentage was 75.7\% and its accuracy, 70.9\%
  \item \cite{Hirano2017} offers preventive therapy for mental healthcare. The users' mental health punctuation significantly improved after using the application. Moreover, the usage rate and the number of suggested actions carried out was high. That indicates that people found the app useful.
  \item \cite{Ring2016} is an agent that responds to users' affective states during virtual therapy sessions. Facial expressions and voices are measured during the session. 70\% of users affirmed that they felt understood by the agent. 50\% of them stated that the agent evoked emotional responses in them during the interactions.
\end{itemize}

\subsubsection{Challenges related to health conversational agents}

Challenges related to health chatbots can be classified into five categories:
\begin{itemize}
  \item Generation and understanding of dialog. Chatbots should be able to understand users in a natural way, taking into account the context of the conversation, voice, ambiguities, etc. and elaborate a question to continue the conversation in a natural way.
  \item Integrate chatbots with other technologies, such as machine learning algorithms to adapt themselves to the user.
  \item Adaptation of chatbots for elder people. They should be simple so that everyone can take advantage of their benefits.
\end{itemize}

\subsubsection{Popularity of chatbots}

Several queries were performed on \textit{scopus.com} to check the number of articles per year related to chatbots and their applications in health. The obtained results were the following ones:
\begin{figure}[H]
  \begin{subfigure}[t]{0.3\textwidth}
    \centering
    \begin{tikzpicture}
      \begin{axis}[title=Result of 'chatbot' query, ytick distance={50}]
        \addplot table {data/chatbot_query_results.dat};
      \end{axis}
    \end{tikzpicture}
  \end{subfigure}
  \hspace{3cm}
  \begin{subfigure}[t]{0.3\textwidth}
    \centering
    \begin{tikzpicture}
      \begin{axis}[title=Result of 'chatbot AND assistant' query, ytick distance={10}]
        \addplot table {data/chatbot_and_assistant_query_results.dat};
      \end{axis}
    \end{tikzpicture}
  \end{subfigure}

  \vspace{1cm}

  \begin{subfigure}[t]{0.3\textwidth}
    \centering
    \begin{tikzpicture}
      \begin{axis}[title=Result of 'chatbot AND health' query, ytick distance={10}]
        \addplot table {data/chatbot_and_health_query_results.dat};
      \end{axis}
    \end{tikzpicture}
  \end{subfigure}
  \hspace{3cm}
  \begin{subfigure}[t]{0.3\textwidth}
    \centering
    \begin{tikzpicture}
      \begin{axis}[title=Result of 'conversational agent' query, ytick distance={50}]
        \addplot table {data/conversational_agent_query_results.dat};
      \end{axis}
    \end{tikzpicture}
  \end{subfigure}
  \caption{Search results of different queries performed in \textit{scopus.com}}
\end{figure}

Although the scale of the different plots is different, we can observe that all of them show a considerable incresing number of articles and papers related to chatbots.

\section{Methodology}

\subsection{Design}

\subsection{Implementation}


\section{Evaluation}

\subsection{Experimental setup}

\subsection{Results}


\section{Discussion}

\section{Conclusions}


\newpage
\bibliographystyle{apacite}
\bibliography{refs}
\end{document}
